\documentclass[10pt,a4paper, french]{article}
%\usepackage{a4wide}
%====================================================================
%  						Liste des Packages
%====================================================================
%--- Packages pour l'écriture ---------------------------------------
\usepackage[T1]{fontenc}
\usepackage{verbatim}
\usepackage[utf8]{inputenc}
\usepackage{babel}
\usepackage{geometry}
\usepackage{indentfirst}
\usepackage{setspace}
%--- Package pour l'insertion de figure -----------------------------
\usepackage{graphicx}
\usepackage[lofdepth,lotdepth]{subfig}
\usepackage{booktabs}
%--- Package pour la création de circuits électronique --------------
\usepackage{tikz}
\usepackage[europeanresistors,americaninductors]{circuitikz}
\newcommand{\speaker}[2] % #1 = name from to[generic,n=#1], #2 = rotation angle
{\draw[thick,rotate=#2] (#1) +(.2,.25) -- +(.7,.75) -- +(.7,-.75) -- +(.2,-.25);}
%--- Package pour référencer proprement -----------------------------
\usepackage{float}
\usepackage{url}
\usepackage{lipsum}
\usepackage{caption}
\usepackage{hyperref}
\usepackage{varioref}
%--- Package pour toutes les unités en physique ---------------------
\usepackage[scientific-notation=true]{siunitx}
%---Tous les Packages qu'il faut pour les maths----------------------
\usepackage{amsmath}
\usepackage{amsfonts}
\usepackage{amsthm}
\usepackage{calrsfs}
\usepackage{mathrsfs}
%--- Configuration des packages -------------------------------------
\geometry{a4paper, total={170mm,257mm}, left=20mm, top=20mm}
\captionsetup[figure]{labelfont=sc} % Ignorer les Warnings ;)
\captionsetup[table]{name=Tableau}
\captionsetup[table]{labelfont=sc} % Ignorer les Warnings ;)
\newcommand{\HRule}{\rule{\linewidth}{0.5mm}}
\newcommand{\ra}[1]{\renewcommand{\arraystretch}{#1}}
%--- Place les Figures en début de page lorsqu'il n'y pas de texte --
\makeatletter
\renewcommand\fps@figure{!htb}
\setlength\@fptop{0pt}
\makeatother
\renewcommand{\floatpagefraction}{.6}% before: .5
\renewcommand{\textfraction}{.15} % before: .2
\renewcommand{\topfraction}{.8}     % before: .7
\renewcommand{\bottomfraction}{.5}  % before: .3
\setcounter{topnumber}{3} % before: 2
\setcounter{bottomnumber}{1} % before: 1
\setcounter{totalnumber}{5} % before: 3

 \begin{document}
%====================================================================
%  							Page de garde
%====================================================================
\begin{titlepage}
	\centering
	\includegraphics[width=0.5\textwidth]{Figures/logo.jpg}\par\vspace{1cm}

	\vspace{2cm}
	{\scshape\LARGE Vibration à N degré de liberté \par}
    \vspace{.5cm}
	\HRule \\[0.7cm]
	{\huge \bfseries Travaux Pratique en Analyse Modale}\\[0.4cm]
	\HRule \\[1.5cm]
	\vspace{1.3cm}
	Auteur:\par
	{\large Maxime \textsc{Glomot}  \par
  \par Barnabé}
	\vspace{.4cm}

	Encadrant:\par
	{\large Olivier \textsc{Richoux} \\
	\vfill

	% \begin{figure}[h]
  %       \centering
  %       \includegraphics[width = 8cm]{Figures/PendulePohl.jpg}
  %       \caption{Pendule de Pohl}
  %       %\label{Figure 1}
  %       \end{figure}
	{\large \today \par}
    Semestre 5 \\
\end{titlepage}

%====================================================================
%   					    Sommaire
%====================================================================
\newpage
%--- Sommaire -------------------------------------------------------
\begin{center}
\renewcommand{\contentsname}{Sommaire}
\tableofcontents
\end{center}
%--- Liste des Figures ----------------------------------------------
\renewcommand*\listfigurename{Liste des Figures}
\listoffigures
%--- Liste des Tableaux ---------------------------------------------
%\renewcommand*\listtablename{Liste des Tableaux}
%\listoftables
%--- Bibliographie --------------------------------------------------
%\begin{thebibliography}{9}
%\bibitem{référenceItem}
%Auteur(s).
%\textit{Titre}.
%Édition, date de parution.
%\end{thebibliography}

%====================================================================
%   					Travail Pratique %====================================================================
\newpage

%\include{Sections/0-Introduction}
%\documentclass[10pt,a4paper, french]{article}
%\usepackage{a4wide}
%====================================================================
%  						Liste des Packages
%====================================================================
%--- Packages pour l'écriture ---------------------------------------
\usepackage[T1]{fontenc}
\usepackage{verbatim}
\usepackage[utf8]{inputenc}
\usepackage{babel}
\usepackage{geometry}
\usepackage{indentfirst}
\usepackage{setspace}
%--- Package pour l'insertion de figure -----------------------------
\usepackage{graphicx}
\usepackage[lofdepth,lotdepth]{subfig}
\usepackage{booktabs}
%--- Package pour la création de circuits électronique --------------
\usepackage{tikz}
\usepackage[europeanresistors,americaninductors]{circuitikz}
\newcommand{\speaker}[2] % #1 = name from to[generic,n=#1], #2 = rotation angle
{\draw[thick,rotate=#2] (#1) +(.2,.25) -- +(.7,.75) -- +(.7,-.75) -- +(.2,-.25);}
%--- Package pour référencer proprement -----------------------------
\usepackage{float}
\usepackage{url}
\usepackage{lipsum}
\usepackage{caption}
\usepackage{hyperref}
\usepackage{varioref}
%--- Package pour toutes les unités en physique ---------------------
\usepackage[scientific-notation=true]{siunitx}
%---Tous les Packages qu'il faut pour les maths----------------------
\usepackage{amsmath}
\usepackage{amsfonts}
\usepackage{amsthm}
\usepackage{calrsfs}
\usepackage{mathrsfs}
%--- Configuration des packages -------------------------------------
\geometry{a4paper, total={170mm,257mm}, left=20mm, top=20mm}
\captionsetup[figure]{labelfont=sc} % Ignorer les Warnings ;)
\captionsetup[table]{name=Tableau}
\captionsetup[table]{labelfont=sc} % Ignorer les Warnings ;)
\newcommand{\HRule}{\rule{\linewidth}{0.5mm}}
\newcommand{\ra}[1]{\renewcommand{\arraystretch}{#1}}
%--- Place les Figures en début de page lorsqu'il n'y pas de texte --
\makeatletter
\renewcommand\fps@figure{!htb}
\setlength\@fptop{0pt}
\makeatother
\renewcommand{\floatpagefraction}{.6}% before: .5
\renewcommand{\textfraction}{.15} % before: .2
\renewcommand{\topfraction}{.8}     % before: .7
\renewcommand{\bottomfraction}{.5}  % before: .3
\setcounter{topnumber}{3} % before: 2
\setcounter{bottomnumber}{1} % before: 1
\setcounter{totalnumber}{5} % before: 3

 \begin{document}
%====================================================================
%  							Page de garde
%====================================================================
\begin{titlepage}
	\centering
	\includegraphics[width=0.5\textwidth]{Figures/logo.jpg}\par\vspace{1cm}

	\vspace{2cm}
	{\scshape\LARGE Vibration à N degré de liberté \par}
    \vspace{.5cm}
	\HRule \\[0.7cm]
	{\huge \bfseries T.P. n\si{\degree} : }\\[0.4cm]
	\HRule \\[1.5cm]
	\vspace{1.3cm}
	Auteur:\par
	{\large Maxime \textsc{Glomot}  \par
  \large Barnabé \textsc{Miesch}}
	\vspace{.4cm}

	Encadrant:\par
	{\large Olivier  \textsc{Richoux} } \\
	\vfill

	\begin{figure}[h]
        \centering
        \includegraphics[width = 8cm]{Figures/PendulePohl.jpg}
        \caption{Pendule de Pohl}
        %\label{Figure 1}
        \end{figure}
	{\large \today \par}
    Semestre 5 \\
\end{titlepage}

%====================================================================
%   					    Sommaire
%====================================================================
\newpage
%--- Sommaire -------------------------------------------------------
\begin{center}
\renewcommand{\contentsname}{Sommaire}
\tableofcontents
\end{center}
%--- Liste des Figures ----------------------------------------------
\renewcommand*\listfigurename{Liste des Figures}
\listoffigures
%--- Liste des Tableaux ---------------------------------------------
%\renewcommand*\listtablename{Liste des Tableaux}
%\listoftables
%--- Bibliographie --------------------------------------------------
%\begin{thebibliography}{9}
%\bibitem{référenceItem}
%Auteur(s).
%\textit{Titre}.
%Édition, date de parution.
%\end{thebibliography}

%====================================================================
%   					Travail Pratique %====================================================================
\newpage


%\include{Sections/3-Annexe}

%====================================================================


\section{Chapître 1 : Vibrations libres et forcées d'une poutre encastrée libre}

\subsection{Introduction}
\subsection{Etude du système en régime libre}





\subsubsection{Excitation par changement de la position d'équilibre}

\begin{figure}
  \includegraphics[scale=0.3]{/path/to/figure}
  \caption{Réponse temporelle (accélèration) de la poutre à son excitation par changement de la position d'équilibre, mesuré à la position de référence $\mathcal{l}$.}
  \label{Figure 1}
\end{figure}

\begin{figure}
  \includegraphics[scale=0.3]{/path/to/figure}
  \caption{Réponse fréquentielle (accélèration) de la poutre à son excitation au milieu de la poutre au marteau d'impact, calculée par FFT à partir du signal représenté \vref{Figure 1}.}
  \label{Figure 2}
\end{figure}




%Combien de modes est-il possible d'observer? Noter leurs caractéristiques.
%Ces modes sont ils harmoniques ?
%Quelle est la précision de l'analyse fréquentielle.
%Comment configurer l'état initial pour n'observer que le premier mode

\begin{figure}
  \includegraphics[scale=0.3]{/path/to/figure}
  \caption{Réponse temporelle (accélèration) de la poutre à son excitation par changement de la position d'équilibre, mesuré à la position de référence $\mathcal{l}$.}
  \label{Figure 3}
\end{figure}

\begin{figure}
  \includegraphics[scale=0.3]{/path/to/figure}
  \caption{Réponse fréquentielle (accélèration) de la poutre à son excitation par changement de la position d'équilibre, au mode $omega_{1,1}$, mesuré à la position de référence $\mathcal{l}$.}
  \label{Figure 4}
\end{figure}

\subsubsection{Excitation impulsionnelle}
% Exciter en son milieu
%Signal
\begin{figure}
  \includegraphics[scale=0.3]{/path/to/figure}
  \caption{Réponse impulsionnelle (accélèration) de la poutre à son excitation au milieu de la poutre au marteau d'impact, mesuré à la position de référence $\mathcal{l}$.}
  \label{Figure 5}
\end{figure}

%Spectre
\begin{figure}
  \includegraphics[scale=0.3]{/path/to/figure}
  \caption{Réponse fréquentielle (accélèration) de la poutre à son excitation au milieu de la poutre par marteau d'impact, calculée par FFT à partir du signal représenté \vref{Figure 5}.}
  \label{Figure 6}
\end{figure}

%Réaliser l'expérience 5 fois: 5 PLOTS Fréquentiels
\begin{figure}
  \includegraphics[scale=0.3]{/path/to/figure}
  \caption{Réponses fréquentielles (accélèration) de la poutre à son excitation au milieu de la poutre au marteau d'impact, calculées par FFT mesuré à la position $\mathcal{l}$.}
  %\label{Figure 7}
\end{figure}
%5 SUBPLOTS
% Dispersion ERROR BAR
\begin{figure}
  \includegraphics[scale=0.3]{/path/to/figure}
  \caption{Réponses fréquentielles (accélèration) de la poutre à son excitation en $\frac{\mathcal{l}}{POSITION}$ au marteau d'impact, calculées par FFT mesuré à la position $\mathcal{l}$.}
  %\label{Figure 7}
\end{figure}

%Autant de figures que de position intéressantes OU bien plusieurs plots sur même figure


\subsection{Etude du système en régime forcé}


Au fréquences pour lesquelles des modes de vibration existent, il y a augmentation de l'amplitude crête du signal. Le premier mode a lieu pour  $10.3 \si{Hz} $ On observe que le 2ème mode a lieu pour $60.3 \si{Hz} $. Le troisième a lieu pour $179 \si{Hsz}$
%Explorer autour des fréquences attendues
\begin{figure}
  \includegraphics[scale=0.3]{/path/to/figure}
  \caption{Réponse temporelle (accélèration) de la poutre à son excitation par pot vibrant à la fréquence $f= \si{Hz}$, mesuré à la position de référence $\mathcal{l}$.}
  %\label{Figure 1}
\end{figure}


%Que se passe-t il à ces fréquences particulières ?

\begin{figure}
  \includegraphics[scale=0.3]{/path/to/figure}
  \caption{Photo de la déformée du premier mode par stroboscope.}
  %\label{Figure 1}
\end{figure}

\begin{figure}
  \includegraphics[scale=0.3]{/path/to/figure}
  \caption{Photo de la déformée du deuxième mode par stroboscope.}
  %\label{Figure 1}
\end{figure}

\begin{figure}
  \includegraphics[scale=0.3]{/path/to/figure}
  \caption{Photo de la déformée du troisième mode par stroboscope.}
  %\label{Figure 1}
\end{figure}

%Observer les lieux des noeuds de déplacement.

\begin{figure}
  \includegraphics[scale=0.3]{/path/to/figure}
  \caption{Réponse temporelle (accélèration) de la poutre à son excitation par pot vibrant autour de la fréquence $f= \si{Hz}$, mesuré sur la position du mode \omega$ $\frac{\mathcal{l}}{POSITION}$.}
  %\label{Figure 1}
\end{figure}


\subsection{Conclusion}



\end{document}

%!TEX root = ../mainTP.tex


\section{Chapître 2 : Oscillations libres et forcées d'un système à un ou deux degrés de liberté}

\subsection{Introduction}
Ce travail pratique a pour object l'étude un système à 1 ou 2 degrés de libertés en flexion illustré par la figure \ref{fig:montage}. Nous étudierons la réponse du système à une excitation impulsionnelle, effectué à l'aide d'un marteau d'impact.

\begin{figure}
    \centering
    \includegraphics[width=200]{Sections/schema_montage.PNG}
    \caption{Schéma du système}
    \label{fig:montage}
\end{figure}


\subsection{Système à 1 degré de liberté}

Le système est composé de deux sous-système masse-ressort à 1 degré de liberté, chacun composé d'une masse montée sur deux plaques en flexions parallèles et supposées respectivement identiques. Dans un premier temps, chaque sous-système est isolé. Pour chaque oscillateur, un accéléromètre est collé sur la face gauche de la masse et l'excitation est effectuée sur la face droite. 

\subsubsection{Analyse temporelle}

L'objectif de cette partie est de déterminer les 2 caractéristiques de chaque oscillateur masse-ressort : \begin{itemize}
    \item $k_1,k_2$, les raideurs des ressorts
    \item $c_1,c_2$, les coefficients d'amortissement
\end{itemize}
\\
L'acquisition des données est réalisée à l'aide du logiciel \textit{Analyseur CTTM} relié à la chaîne de mesure via une \textit{Carte NI}.
\\
Les masses sont pesées avec les lames, dont la masse de ces dernières sont considérées comme non-négligeable : $m_1=419.10^{-3}\pm0,5.10^{-3}kg$ et $m_2=395.10^{-3}\pm0,5.10^{-3}kg$.
\\
Les figures \ref{fig:graph temp 1} et \ref{fig:graph temp 2} illustrent les résultat de cette première série de mesures.

\begin{figure}
    \centering
    \includegraphics{}
    \caption{Réponse impulsionelle du système 1 dans le domaine temporel}
    \label{fig:graph temp 1}
\end{figure}{}

\begin{figure}
    \centering
    \includegraphics{}
    \caption{Réponse impulsionelle du système 2 dans le domaine temporel}
    \label{fig:graph temp 2}
\end{figure}{}

L'allure des courbes expérimentales identique que le système est en régime pseudo-périodique, l'amortissement est donc considéré comme faible. 
Sous l'hypothèse d'amortissement faible, le signal $s(t)$ peut s'écrire sous la forme: \begin{equation}
    s(t)=Ae^{-\zeta\omega_0t}\cos(\omega_dt+\phi)=Ae^{-\delta t}\cos(\omega_dt+\phi)
\end{equation}
Avec $\zeta=\frac{c}{2m\omega_0}$ le taux d'amortissement, $\omega_d=\omega_0\sqrt{1-\zeta^2}$ la pseudo-pulsation et $\omega_0=\sqrt{\frac{k}{m}}$ la pulsation propre du système.
La méthode du décrément logarithmique consiste à effectuer le rapport entre deux maxima ($\cos(\omega_dt+\phi)=1$)afin de calculer le coefficent $\delta$: \begin{equation*}
    \frac{s(t_1)}{s(t_2)}=\frac{A_1}{A_2}=\frac{Ae^{-\delta t_1}}{Ae^{-\delta t_2}}=e^{\delta(t_2-t_1)}
\end{equation*}
\begin{equation*}
    \Rightarrow~\ln{\frac{A_1}{A_2}}=\delta T_d~\Rightarrow~\delta=\frac{1}{T_d}\ln{\frac{A_1}{A_2}} 
\end{equation*}
La pseudo-période $T_d$ ainsi que les amplitudes $A_1$ et $A_2$ sont mesurées, $\delta$ est donc déterminé expérimentalement, ce qui permet d'accéder au coefficient d'amortissement $c$:
\begin{equation*}
    \delta=\zeta\omega_0=\frac{c}{2m}~\Rightarrow~c=\frac{2m}{\delta}
\end{equation*}
D'autre part, connaissant la pseudo période $T_d$ et le décrément logarithmique $\delta$, il est également possible d'accéder à $\omega_0$ puis à la raideur $k$:
\begin{equation*}
    \omega_d=\sqrt{1-\left(\frac{\delta}{\omega_0}\right)^2}=\sqrt{\omega_0^2-\delta^2}~\Rightarrow~\omega_0=\sqrt{\left(\frac{2\pi}{T_d}\right)^2+\delta^2}
\end{equation*}
\begin{equation*}
    \Rightarrow~k=m\omega_0^2
\end{equation*}
\underline{Métrologie} :
Les incertitudes liées aux calculs des différentes caractéristiques de l'oscillateur sont les suivantes :
\begin{equation*}
    \frac{\Delta\delta}{\delta}=\sqrt{\left(\frac{\Delta T_d}{T_d}\right)^2+\left(\frac{\ln{\Delta A_1}}{\ln{A_1}}\right)^2+\left(\frac{\ln{\Delta A_2}}{\ln{A_2}}\right)^2}=
\end{equation*}
\begin{equation*}
    \frac{\Delta c}{c}=\sqrt{\left(\frac{\Delta m}{m}\right)^2+\left(\frac{\Delta\delta}{\delta}\right)^2}=
\end{equation*}
\begin{equation*}
    \frac{\Delta\omega_0}{\omega_0}=\sqrt{\left(\frac{\Delta T_d}{T_d}\right)^2+\left(\frac{\Delta\delta}{\delta}\right)^2}=
\end{equation*}
\begin{equation*}
    \frac{\Delta k}{k}=\sqrt{\left(\frac{\Delta m}{m}\right)^2+\left(\frac{\Delta\omega_0}{\omega_0}\right)^4}=
\end{equation*}
\underline{Applications Numériques} :
\begin{table}[h!]
    \centering
    \begin{tabular}{|c|c|}
        \hline
        Système 1 & Système 2 \\
        \hline    
        $T_{d1}= \pm$ & $T_{d2}= \pm$ \\
        \hline
        $\omega_d_1= \pm$ & $\omega_d_2= \pm$ \\
        \hline
        $\omega_0_1= \pm$ & $\omega_0_2= \pm$ \\
        \hline
        $c_1= \pm$ & $c_2= \pm$ \\
        \hline
        $k_1= \pm$ & $k_2= \pm$ \\
        \hline
    \end{tabular}
    \caption{Caractéristiques expérimentales des deux oscillateurs}
    \label{tab:resultats}
\end{table}{}

COMMENTAIRE COMPARAISON PULSATIONS PROPRES/PSEUDO PULSATIONS

\subsubsection{Analyse fréquentielle}
Dans cette partie, le comportement du système à 1 degré de liberté va être observé dans le domaine spectral.
La FRF (\textit{Fast Fourier Transform}) du signal mesuré est d'abord calculé avec une seule moyenne est d'abord effectuée, puis avec 5 moyennes, puis avec 10 moyennes, pour un temps d'acquisition de $t_max=1s$.

\begin{figure}
    \centering
    \includegraphics{}
    \caption{FRF du système ... 1 moyenne}
    \label{fig:FRF_s..._1moy}
\end{figure}
\begin{figure}
    \centering
    \includegraphics{}
    \caption{FRF du système ... 5 moyenne}
    \label{fig:FRF_s..._1moy}
\end{figure}
\begin{figure}
    \centering
    \includegraphics{}
    \caption{FRF du système ... 10 moyenne}
    \label{fig:FRF_s..._1moy}
\end{figure}

COMMENTAIRE

Ensuite, en variant le pas fréquentiel, la durée d'acquisition varie de $t_{max}=0,2s$ à $t_{max}=5s$.

\begin{figure}
    \centering
    \includegraphics{}
    \caption{FRF du système ... 0,2s d'acquisition}
    \label{fig:FRF_s..._0,2s}
\end{figure}
\begin{figure}
    \centering
    \includegraphics{}
    \caption{FRF du système ... 0,5s d'acquisition}
    \label{fig:FRF_s..._0,5s}
\end{figure}
\begin{figure}
    \centering
    \includegraphics{}
    \caption{FRF du système ... 1s d'acquisition}
    \label{fig:FRF_s..._1s}
\end{figure}
\begin{figure}
    \centering
    \includegraphics{}
    \caption{FRF du système ... 5s d'acquisition}
    \label{fig:FRF_s..._5s}
\end{figure}

COMMENTAIRE
\newpage
\subsection{Système à 2 degrés de liberté}
Maintenant que les deux oscillateurs à 1 degrés de libertés ont été caractérisés. Les deux systèmes sont couplés pour former un système à 2 degrés de liberté. Pour les mesures, un accéléromètre est placés sur chaque face gauche des masses. L'excitation se fait toujours à l'aide du même marteau d'impact.

\subsubsection{Analyse Temporelle}

Dans la même démarche que précédemment, l'objectif est de caractériser l'oscillateur. 
La figure \ref{fig:rep temps 1+2} illustre le résultat de la mesure, l'excitation a été effectuée sur la masse supérieure.

\begin{figure}
    \centering
    \includegraphics{}
    \caption{Réponse impulsionelle du système 1+2 dans le domaine temporel}
    \label{fig:rep temps 1+2}
\end{figure}

COMMENTAIRE
Les deux 

\subsubsection{Analyse Fréquentielle}

\subsection{Conclusion}



%\documentclass[10pt,a4paper, french]{article}
%\usepackage{a4wide}
%====================================================================
%  						Liste des Packages
%====================================================================
%--- Packages pour l'écriture ---------------------------------------
\usepackage[T1]{fontenc}
\usepackage{verbatim}
\usepackage[utf8]{inputenc}
\usepackage{babel}
\usepackage{geometry}
\usepackage{indentfirst}
\usepackage{setspace}
%--- Package pour l'insertion de figure -----------------------------
\usepackage{graphicx}
\usepackage[lofdepth,lotdepth]{subfig}
\usepackage{booktabs}
%--- Package pour la création de circuits électronique --------------
\usepackage{tikz}
\usepackage[europeanresistors,americaninductors]{circuitikz}
\newcommand{\speaker}[2] % #1 = name from to[generic,n=#1], #2 = rotation angle
{\draw[thick,rotate=#2] (#1) +(.2,.25) -- +(.7,.75) -- +(.7,-.75) -- +(.2,-.25);}
%--- Package pour référencer proprement -----------------------------
\usepackage{float}
\usepackage{url}
\usepackage{lipsum}
\usepackage{caption}
\usepackage{hyperref}
\usepackage{varioref}
%--- Package pour toutes les unités en physique ---------------------
\usepackage[scientific-notation=true]{siunitx}
%---Tous les Packages qu'il faut pour les maths----------------------
\usepackage{amsmath}
\usepackage{amsfonts}
\usepackage{amsthm}
\usepackage{calrsfs}
\usepackage{mathrsfs}
%--- Configuration des packages -------------------------------------
\geometry{a4paper, total={170mm,257mm}, left=20mm, top=20mm}
\captionsetup[figure]{labelfont=sc} % Ignorer les Warnings ;)
\captionsetup[table]{name=Tableau}
\captionsetup[table]{labelfont=sc} % Ignorer les Warnings ;)
\newcommand{\HRule}{\rule{\linewidth}{0.5mm}}
\newcommand{\ra}[1]{\renewcommand{\arraystretch}{#1}}
%--- Place les Figures en début de page lorsqu'il n'y pas de texte --
\makeatletter
\renewcommand\fps@figure{!htb}
\setlength\@fptop{0pt}
\makeatother
\renewcommand{\floatpagefraction}{.6}% before: .5
\renewcommand{\textfraction}{.15} % before: .2
\renewcommand{\topfraction}{.8}     % before: .7
\renewcommand{\bottomfraction}{.5}  % before: .3
\setcounter{topnumber}{3} % before: 2
\setcounter{bottomnumber}{1} % before: 1
\setcounter{totalnumber}{5} % before: 3

 \begin{document}
%====================================================================
%  							Page de garde
%====================================================================
\begin{titlepage}
	\centering
	\includegraphics[width=0.5\textwidth]{Figures/logo.jpg}\par\vspace{1cm}

	\vspace{2cm}
	{\scshape\LARGE Vibration à N degré de liberté \par}
    \vspace{.5cm}
	\HRule \\[0.7cm]
	{\huge \bfseries T.P. n\si{\degree} : }\\[0.4cm]
	\HRule \\[1.5cm]
	\vspace{1.3cm}
	Auteur:\par
	{\large Maxime \textsc{Glomot}  \par
  \large Barnabé \textsc{}}
	\vspace{.4cm}

	Encadrant:\par
	{\large \textsc{Ablitzer} Fréderic} \\
	\vfill

	\begin{figure}[h]
        \centering
        \includegraphics[width = 8cm]{Figures/PendulePohl.jpg}
        \caption{Pendule de Pohl}
        %\label{Figure 1}
        \end{figure}
	{\large \today \par}
    Semestre 5 \\
\end{titlepage}

%====================================================================
%   					    Sommaire
%====================================================================
\newpage
%--- Sommaire -------------------------------------------------------
\begin{center}
\renewcommand{\contentsname}{Sommaire}
\tableofcontents
\end{center}
%--- Liste des Figures ----------------------------------------------
\renewcommand*\listfigurename{Liste des Figures}
\listoffigures
%--- Liste des Tableaux ---------------------------------------------
%\renewcommand*\listtablename{Liste des Tableaux}
%\listoftables
%--- Bibliographie --------------------------------------------------
%\begin{thebibliography}{9}
%\bibitem{référenceItem}
%Auteur(s).
%\textit{Titre}.
%Édition, date de parution.
%\end{thebibliography}

%====================================================================
%   					Travail Pratique %====================================================================
\newpage


%\include{Sections/3-Annexe}

%====================================================================



\section{Chapître 3 : Etude des vibration engendrées par un touret à meuler}

\subsection{Introduction}
\subsection{Etude des modes propres de vibration}
\subsection{Mise en œuvre d'un absorbeur dynamique}
\subsection{Conclusion}




\end{document}

%\include{Sections/4-Chap4}
%\include{Sections/5-Conclusion}
%\include{Sections/3-Annexe}

%====================================================================
\end{document}
