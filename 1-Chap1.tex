\documentclass[10pt,a4paper, french]{article}
%\usepackage{a4wide}
%====================================================================
%  						Liste des Packages
%====================================================================
%--- Packages pour l'écriture ---------------------------------------
\usepackage[T1]{fontenc}
\usepackage{verbatim}
\usepackage[utf8]{inputenc}
\usepackage{babel}
\usepackage{geometry}
\usepackage{indentfirst}
\usepackage{setspace}
%--- Package pour l'insertion de figure -----------------------------
\usepackage{graphicx}
\usepackage[lofdepth,lotdepth]{subfig}
\usepackage{booktabs}
%--- Package pour la création de circuits électronique --------------
\usepackage{tikz}
\usepackage[europeanresistors,americaninductors]{circuitikz}
\newcommand{\speaker}[2] % #1 = name from to[generic,n=#1], #2 = rotation angle
{\draw[thick,rotate=#2] (#1) +(.2,.25) -- +(.7,.75) -- +(.7,-.75) -- +(.2,-.25);}
%--- Package pour référencer proprement -----------------------------
\usepackage{float}
\usepackage{url}
\usepackage{lipsum}
\usepackage{caption}
\usepackage{hyperref}
\usepackage{varioref}
%--- Package pour toutes les unités en physique ---------------------
\usepackage[scientific-notation=true]{siunitx}
%---Tous les Packages qu'il faut pour les maths----------------------
\usepackage{amsmath}
\usepackage{amsfonts}
\usepackage{amsthm}
\usepackage{calrsfs}
\usepackage{mathrsfs}
%--- Configuration des packages -------------------------------------
\geometry{a4paper, total={170mm,257mm}, left=20mm, top=20mm}
\captionsetup[figure]{labelfont=sc} % Ignorer les Warnings ;)
\captionsetup[table]{name=Tableau}
\captionsetup[table]{labelfont=sc} % Ignorer les Warnings ;)
\newcommand{\HRule}{\rule{\linewidth}{0.5mm}}
\newcommand{\ra}[1]{\renewcommand{\arraystretch}{#1}}
%--- Place les Figures en début de page lorsqu'il n'y pas de texte --
\makeatletter
\renewcommand\fps@figure{!htb}
\setlength\@fptop{0pt}
\makeatother
\renewcommand{\floatpagefraction}{.6}% before: .5
\renewcommand{\textfraction}{.15} % before: .2
\renewcommand{\topfraction}{.8}     % before: .7
\renewcommand{\bottomfraction}{.5}  % before: .3
\setcounter{topnumber}{3} % before: 2
\setcounter{bottomnumber}{1} % before: 1
\setcounter{totalnumber}{5} % before: 3

 \begin{document}
%====================================================================
%  							Page de garde
%====================================================================
\begin{titlepage}
	\centering
	\includegraphics[width=0.5\textwidth]{Figures/logo.jpg}\par\vspace{1cm}

	\vspace{2cm}
	{\scshape\LARGE Vibration à N degré de liberté \par}
    \vspace{.5cm}
	\HRule \\[0.7cm]
	{\huge \bfseries T.P. n\si{\degree} : }\\[0.4cm]
	\HRule \\[1.5cm]
	\vspace{1.3cm}
	Auteur:\par
	{\large Maxime \textsc{Glomot}  \par
  \large Barnabé \textsc{}}
	\vspace{.4cm}

	Encadrant:\par
	{\large \textsc{Ablitzer} Fréderic} \\
	\vfill

	\begin{figure}[h]
        \centering
        \includegraphics[width = 8cm]{Figures/PendulePohl.jpg}
        \caption{Pendule de Pohl}
        %\label{Figure 1}
        \end{figure}
	{\large \today \par}
    Semestre 5 \\
\end{titlepage}

%====================================================================
%   					    Sommaire
%====================================================================
\newpage
%--- Sommaire -------------------------------------------------------
\begin{center}
\renewcommand{\contentsname}{Sommaire}
\tableofcontents
\end{center}
%--- Liste des Figures ----------------------------------------------
\renewcommand*\listfigurename{Liste des Figures}
\listoffigures
%--- Liste des Tableaux ---------------------------------------------
%\renewcommand*\listtablename{Liste des Tableaux}
%\listoftables
%--- Bibliographie --------------------------------------------------
%\begin{thebibliography}{9}
%\bibitem{référenceItem}
%Auteur(s).
%\textit{Titre}.
%Édition, date de parution.
%\end{thebibliography}

%====================================================================
%   					Travail Pratique %====================================================================
\newpage


%\include{Sections/3-Annexe}

%====================================================================


\section{Chapître 1 : Vibrations libres et forcées d'une poutre encastrée libre}

\subsection{Introduction}
\subsection{Etude du système en régime libre}
\subsubsection{Excitation par changement de la position d'équilibre}
\subsubsection{Excitation impulsionnelle}
\subsection{Etude du système en régime forcé}
\subsection{Conclusion}



\end{document}
